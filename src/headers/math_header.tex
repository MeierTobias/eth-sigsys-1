%Mathematik-Pakete
\usepackage{amsmath, amstext, amssymb, mathtools, esint, polynom, trfsigns, pgfplots}
%Spec. version of pgfplots (rel. new version 2023)
\pgfplotsset{compat=1.18}
\usepackage{bm}
\allowdisplaybreaks %Seitenumbruch in align-Umgebung erlauben
%

%Definition der Umgebung "example"
\newenvironment {example}
{\begin{itshape} \begin{small}}
			{\end{small} \end{itshape}}
%				
%Definition der Umgebung "annotation"		
\newenvironment {annotation}[1]
{\begin{itshape} \begin{small} \textbf{#1} \begin{itemize}}
				{\end{itemize} \end{small} \end{itshape}}
%				
%Definition der Umgebung "eq"
\newenvironment {eq}
{\begin{equation*}}
		{\end{equation*}}
%
% Don't know what this does
\providecommand{\diff}{\mathop{} \! \mathrm{d}}
\DeclareMathOperator{\rot}{rot}
\DeclareMathOperator{\divg}{div}

% Block-diagram
% A very nice example can be found here: https://tex.stackexchange.com/questions/175969/block-diagrams-using-tikz
\usepackage{tikz}
\usetikzlibrary{shapes,arrows,positioning,calc}
\tikzset{
	block/.style = {draw, fill=white, rectangle, minimum height=3em, minimum width=3em},
	tmp/.style  = {coordinate},
	sum/.style= {draw, fill=white, circle, node distance=1cm},
	input/.style = {coordinate},
	output/.style= {coordinate},
	pinstyle/.style = {pin edge={to-,thin,black}}
}