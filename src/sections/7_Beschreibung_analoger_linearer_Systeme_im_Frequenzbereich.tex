\section{Analoge lineare Systeme im Frequenzbereich}
\subsection{Eigenfunktionen}

\begin{center}
    \begin{tikzpicture}[auto, node distance=2cm,>=latex']
        \node [input, name=rinput] (rinput) {};
        \node [block, right of=rinput] (h1) {$H$};
        \node [output, right of=h1, node distance=2cm] (output) {};
        \draw [->] (rinput) -- node{$x(t)$} (h1);
        \draw [->] (h1) -- node [name=y] {$y(t)$}(output);
    \end{tikzpicture}
\end{center}

Die Funktionen $e^{2\pi{}ift}$ sind Eigenfunktionen des Systems H, die zugehörigen Eigenwerte sind $\hat{h}(f)$. Der Frequenzgang des LTI-Systems ist gleich der Fouriertransformation $\hat{h}(f)$ der Impulsantwort $h(t)$.

\begin{align*}
    Hx               & =\lambda{}x                  \\
    He^{2\pi{}if_0t} & =\hat{h}(f_0)e^{2\pi{}if_0t}
\end{align*}

\textbf{Analogie zu Eigenvektoren und Eigenwerten von Matrizen.}
Es sei $H \in \mathbb{C}^{n\times{}n}$ eine Matrix mit der nachfolgenden Eigenwertzerlegung.
\begin{equation*}
    H=U\Sigma{}V^H
\end{equation*}
Wenn H normal ist ($HH^H=H^H H$):
\begin{equation*}
    H=U\Sigma{}U^H=\sum_{i=1}^{n}u_i\lambda_i u_i^H=\color{sectionColor}{\sum_{i=1}^{n}\lambda_i u_i u_i^H}\\
\end{equation*}
Dabei ist
\begin{equation*}
    UU^H = U^H U=I_n \Leftrightarrow \langle{}u_i,u_j\rangle{}=
    \begin{cases}
        1, & i=j           \\
        0, & \text{sonst.}
    \end{cases}
\end{equation*}
Wird die Matrix mit einem Eigenvektor "angeregt", so erhalten wir % chktex 18
\begin{align*}
    Hu_\ell & ={\color{sectionColor}{\left(\sum_{i=1}^n\lambda_i u_i u_i^H\right)}}u_\ell=\sum_{i=1}^n\lambda_i u_i\underbrace{(u_i^H u_\ell)}_{=\delta_{i-\ell}} \\
    Hu_\ell & =\lambda_\ell{}u_\ell
\end{align*}

Wenn wir den Vektor $x$ als Linearkombination der Basisvektoren $u_i$ schreiben
\begin{equation*}
    x=\sum_{i=1}^{n}c_i u_i, \text{mit Koeffizienten} c_i=\langle{}x,u_i\rangle
\end{equation*}
Dann erhält man für das Ausgangssignal
\begin{align*}
    Hx & ={\color{sectionColor}{\left(\sum_{i=1}^n\lambda_i u_i u_i^H\right)}} \sum_{j=1}^n c_j u_j \\
       & =\sum_{i=1}^n\sum_{j=1}^n\lambda_i c_j u_i\underbrace{u_i^H u_j}_{=\delta_{i-j}}           \\
       & =\sum_{i=1}^n\lambda_i c_i u_i
\end{align*}
Diese Darstellung entspricht der zuvor abgeleiteten Beziehung für zeitkontinuierliche Systeme
\begin{equation*}
    (Hx)(t)=\int_{-\infty}^\infty\underbrace{\hat{h}(f)}_{\lambda_i}\underbrace{\hat{x}(f)}_{c_i=\langle{}x,u_i\rangle}\underbrace{e^{2\pi ift}}_{u_i}df
\end{equation*}

\subsubsection{Kaskadierung}
\begin{center}
    \begin{tikzpicture}[auto, node distance=2cm,>=latex']
        \node [input, name=rinput] (rinput) {};
        \node [block, right of=rinput] (h1) {$H_1$};
        \node [block, right of=h1] (h2) {$H_2$};
        \node [output, right of=h2, node distance=2cm] (output) {};
        \draw [->] (rinput) -- node{$x(t)$} (h1);
        \draw [->] (h1) -- node{$y_1(t)$} (h2);
        \draw [->] (h2) -- node [name=y] {$y(t)$}(output);
    \end{tikzpicture}
\end{center}
\begin{align*}
    h(t)    & =(h_1*h_2)(t)       \\
    \hat{h} & =\hat{h_1}\hat{h_2}
\end{align*}

\subsubsection{Weitere spezielle Eingangssignale}
\textbf{Allgemeine Schwingungen:}
\begin{align*}
    x(t)   & =e^{st}             \\
    \intertext{mit}
    s      & =\sigma + i2\pi f_0 \\
    \Re(s) & = \sigma
    \begin{cases}
        <0, & \text{zeitlich abklingende Einhüllende} \\
        =0, & \text{zeitlich konstante Einhüllende}   \\
        >0, & \text{zeitlich anklingende Einhüllende}
    \end{cases}
\end{align*}
Das zugehörige Ausgangssignal lässt sich in diesem Fall schreiben als
\begin{equation*}
    y(t)=(x*h)(t)=\int_{-\infty}^\infty h(\tau)e^{s(t-\tau)}d\tau=e^{st}\underbrace{\int_{-\infty}^\infty h(\tau)e^{-s\tau}d\tau}_{=:H(s)}
\end{equation*}
wobei $H(s)$ die Laplacetransformierte von $h(t)$ genannt wird.

\bigskip

\textbf{Sinusförmiges Eingangssignal}
\begin{equation*}
    x(t)=\cos(2\pi f_0t+\varphi_0)=\frac{1}{2}e^{2\pi if_0t}e^{i\varphi_0}+\frac{1}{2}e^{-2\pi if_0t}e^{-i\varphi_0}
\end{equation*}
Ausgangssignal:
\begin{equation*}
    y(t)=\frac12\hat{h}(f_0)e^{2\pi if_0t}e^{i\varphi_0}+\frac12\hat{h}(-f_0)e^{-2\pi if_0t}e^{-i\varphi_0}
\end{equation*}
Für $h(t)\in\mathbb{R}$ gilt $\hat{h}(f)=\hat{h}^*(-f)$ und somit
\begin{equation*}
    y(t) = |\hat{h}(f_0)|\cos(2\pi f_0t+\varphi_0+\arg\hat{h}(f_0))
\end{equation*}
$|\hat{h}(f)|$ gibt die Änderung der Amplitude und $\arg(\hat{h}(f))$ die Änderung der Phase gegenüber
dem Eingangssignal an.

\subsection{Die Fouriertransformation}
Definition:
\begin{equation*}
    \hat{x}(f):=(\mathcal{F}x)(f):=\int_{-\infty}^\infty x(t)e^{-2\pi ift}dt
\end{equation*}
Rücktransformation
\begin{equation*}
    x(t):=(\mathcal{F}^{-1}\hat{x})(t):=\int_{-\infty}^\infty\hat{x}(f)e^{2\pi ift}df
\end{equation*}

\smallskip

\textbf{Riemann-Lebesgue Lemma}

Es sei $x$ ein absolut integrierbares Signal, d.h. $x \in L^1$. Dann ist $(\mathcal{F}x)(f)=\hat{x}(f)$ stetig und es gilt $\lim_{|f|\to\infty}(\mathcal{F}x)(f)=0$.

\subsubsection{Fouriertransformation verallgemeinerter Funktionen}
oder Funktionen die weder in $L^1$ noch in $L^2$ sind (z.B. $x(t)=1$ oder $x(t)=e^{i2\pi at}$).
\\
\textbf{Plancherelsche Identität}
Es sei $x,y \in L^2$
\begin{align*}
    \int_{-\infty}^\infty x(t)y^*(t)dt & =\int_{-\infty}^\infty\hat{x}(f)\hat{y}^*(f)df \\
    \langle x,y\rangle                 & = \langle\hat{x},\hat{y}\rangle
\end{align*}
Diese Gleichung dient als Grundlage zur Erweiterung des bisherigen Testfunktionenraums $\mathcal{D}$ auf den linearen Raum $\mathcal{S}$ aller beliebig oft differenzierbaren Funktionen, die gemeinsam mit jeder Ableitung für $|u| \rightarrow \infty$ stärker als jede Potenz von $1/|u|$ gegen Null gehen. Der resultierende Testfunktionenraum, genannt $\mathcal{S}$, enthält alle bisher betrachteten Testfunktionen und hat die Eigenschaft, dass für jede Funktion $\varphi \in \mathcal{S}$ auch $\hat{\varphi} = \mathcal{F}\varphi \in \mathcal{S}$. Ein Funktional auf $\mathcal{S}$ wird als temperierte verallgemeinerte Funktion bezeichnet.
Somit kann gezeigt werden, dass gilt:
\begin{align*}
    \ell_x(\hat{\varphi})      & =\ell_{\hat{x}}(\varphi)   \\
    \ell_x(\mathcal{F}\varphi) & =\ell_{\mathcal{F}x}(\ell) \\
    \ell(\mathcal{F}\varphi)   & =:\mathcal{F}\ell(\varphi)
\end{align*}
\begin{equation*}
    \mathcal{F}\ell_x(\varphi):=\ell_x(\mathcal{F}\varphi)=\ell_{\mathcal{F}x}(\varphi)
\end{equation*}
Folgende Eigenschaften gelten auch für $\varphi \in \mathcal{S}$:
\begin{align*}
     & \text{Additivität}     &  & (\ell+\widetilde{\ell})(\varphi):=\ell(\varphi)+\widetilde{\ell}(\varphi) \\
     & \text{Homogenität}     &  & (\alpha\ell)(\varphi):=\alpha\ell(\varphi)                                \\
     & \text{Zusammensetzung} &  & (\ell\circ g)(\varphi):=\frac{1}{|a|}\ell(\psi),                          \\
     &                        &  & \psi(t)=\varphi\!\left(\frac{t-b}{a}\right),\quad g(t)=at+b               \\
     & \text{Multiplikation}  &  & (\psi\ell)(\varphi):=\ell(\psi\varphi)
\end{align*}
wobei für die Multiplikation noch die Einschränkung hinzukommt, dass $\phi(t)$ für $|t| \rightarrow \infty$ nicht allzu stark wachsen darf, damit auch $\psi(t)\varphi(t) \in \mathcal{S} $.
Des weiteren bleibt für die Differentiation gültig:
\begin{align*}
     & \text{Ableitung}    &  & D\ell(\varphi)=\ell'(\varphi):=-\ell(\varphi')                           \\
     & \text{Additivität}  &  & (\ell+\tilde{\ell})'(\varphi):=\ell'(\varphi)+\widetilde{\ell}'(\varphi) \\
     & \text{Homogenität}  &  & (\alpha\ell)'(\varphi):=\alpha\ell'(\varphi)                             \\
     & \text{Produktregel} &  & (\psi\ell)'=\psi'\ell+\psi\ell'
\end{align*}

\textbf{Vorgehen:}

Auf verallgemeinerte Funktionen $\ell$ anwenden damit auch nicht absolut integrierbare Funktionen verwendet werden können.
\begin{align*}
    y               & =\mathcal{F}x                                            \\
                    & \downarrow \ell                                          \\
    \ell_y(\varphi) & =\ell_{\mathcal{F}x}(\varphi)=\ell_x(\mathcal{F}\varphi)
\end{align*}

\textbf{Einige Fouriertransformationspaare:}
\begin{align*}
    \delta(t) \;                                  & \laplace\; 1                                                        \\
    sign(t) \;                                    & \laplace\; \frac{1}{\pi if}                                         \\
    \Sigma(t) \;                                  & \laplace\; \frac{1}{2\pi if}+\frac{1}{2}\delta(f)                   \\
    e^{2\pi iat} \;                               & \laplace\; \delta(f-a)                                              \\
    \sin(2\pi f_0t) \;                            & \laplace\; \frac{i}{2}\left(\delta(f+f_0)-\delta(f-f_0)\right)      \\
    \cos(2\pi f_0t) \;                            & \laplace\; \frac{1}{2}\left(\delta(f+f_0)+\delta(f-f_0)\right)      \\
    \sum_{k=-\infty}^\infty c_k e^{2\pi ikt/T} \; & \laplace\; \sum_{k=-\infty}^\infty c_k\delta\left(f-\frac kT\right)
\end{align*}
Es folgen weitere Beziehungen, die für verallgemeinerte Funktionen ihre Gültigkeit bewahren:
\begin{align*}
    (\mathcal{F}^2x)(f)                               & = x(-f)                                                                     \\
    \left({(\mathcal{F}^{-1})}^2\widehat{x}\right)(t) & = \hat(-t)                                                                  \\
    x(t-t_{0}) \;                                     & \laplace\; e^{-2\pi ift_0}\hat{x}(f)                                        \\
    x(at+b) \;                                        & \laplace\; \frac{1}{|a|}\int_{-\infty}^{\infty}x(t')e^{-2\pi if(t'-b)/a}dt' \\
                                                      & =\frac1{|a|}e^{i2\pi fb/a}\hat{x}(f/a)                                      \\
    e^{2\pi if_{0}t}x(t) \;                           & \laplace\; \hat{x}(f-f_0)
\end{align*}

\textbf{Die Fouriertransformation als Rotation in der Zeit-Frequenz-Ebene:}
\begin{center}
    \includegraphics[width=0.6\linewidth]{img/7.2.1_FT_Rotation.png}
\end{center}
\begin{align*}
    x(t) \;       & \laplace\; \hat{x}(f)  \\
    \hat{x}(t) \; & \laplace\; x(-f)       \\
    x(-t) \;      & \laplace\; \hat{x}(-f) \\
    \hat{x}(-t)\; & \laplace\; x(f)
\end{align*}

\textbf{Differentiation}
\begin{equation}
    \mathcal{F}x'=2\pi if\mathcal{F}x=2\pi if\hat{x}
\end{equation}

\textbf{Faltung verallgemeinerter Funktionen}
\begin{equation}
    (\mathcal{F}(x*y))(f) = \hat{x}(f)\hat{y}(f)
\end{equation}

\subsubsection{Periodische Signale an LTI-Systemen}
Ein LTI-System antwortet auf ein periodisches Eingangssignal
\begin{equation*}
    x(t)=\sum_{k=-\infty}^{\infty}c_k e^{2\pi it\frac{k}{T}}
\end{equation*}
mit dem Ausgangssignal
\begin{equation*}
    y(t)=h(t)*x(t)=\sum_{k=-\infty}^{\infty}\hat{h}\Biggl(\frac{k}{T}\Biggr)c_k e^{2\pi it\frac{k}{T}}
\end{equation*}
Beweis der Periodizität:
\begin{equation*}
    y(t-T)=\cdots=y(t)
\end{equation*}

\textbf{Poissonsche Summenformel}
\begin{equation*}
    \sum_{k=-\infty}^{\infty}h(t-kT)=\frac{1}{T}\sum_{k=-\infty}^{\infty}\hat{h}\left(\frac{k}{T}\right)e^{2\pi ikt/T}
\end{equation*}
Daraus kann abgeleitet werden, dass die Fouriertransformierte eines Impulskammes wieder ein Impulskamm ist.
\begin{equation*}
    \sum_{k=-\infty}^{\infty}\delta(t-kT) \: \laplace \: \frac{1}{T}\sum_{k=-\infty}^{\infty}\delta\left(f-\frac{k}{T}\right)
\end{equation*}

\subsection{Anw.\ der Fouriertransformation auf LTI-Systeme}

\subsubsection{Idealisierter Tiefpassfilter}
\textbf{Allpass}
Für ein verzerrungsfreies System gilt: (reine Verstärkung und Phasenverschiebung)
\begin{equation*}
    y(t) = kx(t-t_0)
\end{equation*}
Die Fouriertransformierte der Impulsantwort zeigt den Amplituden und Phasengang:
\begin{equation*}
    \hat{h}(t) = k e^{-2\pi ift_0} = |\hat{h}(f)|e^{i\varphi(f)}
\end{equation*}
mit
\begin{align*}
     & |\hat{h}(f)| = |k|            \\
     & \varphi(f)   = -2\pi ft_0     \\
     & h(t)         = k\delta(t-t_0)
\end{align*}

\begin{center}
    \begin{tikzpicture}
        \begin{axis}[
                width=0.5\linewidth,
                unit vector ratio={1 1},
                axis x line=left,
                axis y line=middle,
                xmin=-2,
                xmax=2,
                ymin=0,
                ymax=2,
                xlabel={$f$},
                ylabel={$|\hat{h}(f)|$},
                xtick=\empty,
                ytick={0},
                mark=none,
            ]
            \addplot [blue]
            coordinates {
                    (\pgfkeysvalueof{/pgfplots/xmin},1)
                    (\pgfkeysvalueof{/pgfplots/xmax},1)
                };
        \end{axis}
    \end{tikzpicture}
    \quad
    \begin{tikzpicture}
        \begin{axis}[
                width=0.5\linewidth,
                unit vector ratio={1 1},
                axis x line=middle,
                axis y line=middle,
                xmin=-2,
                xmax=2,
                ymin=-2,
                ymax=2,
                xlabel={$f$},
                ylabel={$\varphi (f)$},
                xtick=\empty,
                ytick={0},
                mark=none,
            ]
            \addplot [blue]
            coordinates {
                    (\pgfkeysvalueof{/pgfplots/xmin},\pgfkeysvalueof{/pgfplots/xmax})
                    (\pgfkeysvalueof{/pgfplots/xmax},\pgfkeysvalueof{/pgfplots/xmin})
                };
        \end{axis}
    \end{tikzpicture}
\end{center}

Eine scharfe Bandbegrenzung verursacht nachfolgende Verzerrung:

\begin{tabular}[c]{@{}p{0.5\linewidth}p{0.4\linewidth}@{}}
    \begin{minipage}[top]{\linewidth}
        \includegraphics*[width=\linewidth]{img/7.3.1_TP_f.png}
    \end{minipage}
     &
    \begin{minipage}[top]{\linewidth}
        {
            \begin{align*}
                |\hat{h}(f)| & =
                \begin{cases}
                    1, & |f|\leq f_g  \\
                    0, & \text{sonst}
                \end{cases}           \\
                \varphi(f)   & = -2\pi ft_0
            \end{align*}
        }
    \end{minipage}
    \\
    \begin{minipage}[top]{\linewidth}
        \includegraphics*[width=\linewidth]{../img/7.3.1_TP_t.png}
    \end{minipage}
     &
    \begin{minipage}[top]{\linewidth}
        \begin{equation*}
            h(t) = \frac{\sin(2\pi f_g(t-t_0))}{\pi (t-t_0)}
        \end{equation*}
    \end{minipage}
\end{tabular}

Dieses System ist nicht kausal. Ferner ist $h(t)$ nicht absolut integrierbar, denn wenn $h(t)$ absolut integrierbar wäre, dann wäre $\hat{h}(f)$ stetig (siehe Riemann-Lebesgue Lemma). Unstetigkeiten im Frequenzgang ergeben daher immer eine Impulsantwort,
die nicht absolut integrierbar ist.

Die Sprungantwort ist:
\begin{equation*}
    a(t) = \int_{-\infty}^{\infty}\sigma(t-\tau)h(\tau)d\tau=\int_{-\infty}^{t}h(\tau)d\tau
\end{equation*}

\begin{center}
    \includegraphics*[width=0.9\linewidth]{../img/7.3.1_stepResponse.png}
\end{center}

\begin{itemize}
    \item Tangente mit maximaler Steigung $2f_g$ bei $t=t_0$
    \item Die maximale Steigung ist proportional zur Grenzfrequenz $fg$.
    \item Ein Tiefpass mit höherer Grenzfrequenz kann einer Signaländerung rascher folgen.
\end{itemize}


Um den idealisierten Tiefpass \textbf{kausal} zu machen, muss $h(t)$ nach rechts verschoben werden und dann für $t < 0$ zu Null gesetzt werden. Die Impulsantwort sieht dann
folgendermassen aus
\begin{equation*}
    h_{kaus}(t)=h_{id}(t-t_1)\sigma(t)
\end{equation*}
\begin{center}
    \includegraphics[width=0.7\linewidth]{../img/7.3.1_TP_kaus_t.png}
\end{center}
Mit der Fouriertransformierten
\begin{equation*}
    \hat{h}_{kaus}(f) = (e^{-2\pi ift_1}\hat{h}_{id}(f))*\frac{1}{2\pi if}+\frac{1}{2}e^{-2\pi ift_1}\hat{h}_{id}(f)
\end{equation*}
\begin{tabular}[c]{@{}p{0.5\linewidth}p{0.45\linewidth}@{}}
    \begin{minipage}[top]{\linewidth}
        \includegraphics*[width=\linewidth]{../img/7.3.1_TP_kaus_f.png}
    \end{minipage}
     &
    \begin{minipage}[top]{\linewidth}
        Es kommt zur Überhöhung im Frequenzgang bei $\pm f_g$.
    \end{minipage}
\end{tabular}

Um den idealisierten Tiefpass stabil zu machen, kann wie folgt vorgegangen werden
\begin{center}
    \includegraphics*[width=\linewidth]{../img/7.3.1_TP_stab.png}
\end{center}
Dabei erhält man Stabilität aus der Tatsache, dass
\begin{equation*}
    \left(\hat{h}_{id}*\hat{h}_{ge}\right)(f) \: \laplace \: \sim \frac{1}{t}\frac{1}{t} \Rightarrow h\in L^1
\end{equation*}

\subsubsection{Bandbegrenzte Signale}
Dei Bandbreite des Signals $x$ ist das kleinste $W$, so dass
\begin{equation*}
    (x*h_{TP,W})(t)=x(t)
\end{equation*}
mit
\begin{equation*}
    \hat{h}_{TP,W}(f)=
    \begin{cases}
        1, & |f| \leq W \\
        0, & sonst.
    \end{cases}
\end{equation*}

\textbf{Bernstein-Ungleichung}
Die Bernsteinsche Ungleichung gibt der Intuition Ausdruck, dass Signale mit kleiner
Bandbreite im Zeitbereich nur langsame Änderungen aufweisen können.

Wenn $x(t)$ in der Form
\begin{equation*}
    x(t)=\int_{-W}^{W}g(f)e^{2\pi ift}df,\quad\text{für alle}t\in\mathbb{R}
\end{equation*}
dargestellt werden kann, für eine integrierbare Funktion $g$, d.h. $g \in L^1$, dann gilt
\begin{equation*}
    \left|\frac{dx(t)}{dt}\right|\leq4\pi W\sup_{\tau\in\mathbb{R}}|x(\tau)|,\quad\text{für alle}t\in\mathbb{R}.
\end{equation*}