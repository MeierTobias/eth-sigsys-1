\section{Analoge lineare Systeme im Frequenzbereich}
\subsection{Eigenfunktionen}

\begin{center}
    \begin{tikzpicture}[auto, node distance=2cm,>=latex']
        \node [input, name=rinput] (rinput) {};
        \node [block, right of=rinput] (h1) {$H$};
        \node [output, right of=h1, node distance=2cm] (output) {};
        \draw [->] (rinput) -- node{$x(t)$} (h1);
        \draw [->] (h1) -- node [name=y] {$y(t)$}(output);
    \end{tikzpicture}
\end{center}

Die Funktionen $e^{2\pi{}ift}$ sind Eigenfunktionen des Systems H, die zugehörigen Eigenwerte sind $\hat{h}(f)$. Der Frequenzgang des LTI-Systems ist gleich der Fouriertransformation $\hat{h}(f)$ der Impulsantwort $h(t)$.

\begin{align*}
    Hx&=\lambda{}x \\
    He^{2\pi{}if_0t}&=\hat{h}(f_0)e^{2\pi{}if_0t}
\end{align*}

\textbf{Analogie zu Eigenvektoren und Eigenwerten von Matrizen.}
Es sei $H \in \mathbb{C}^{n\times{}n}$ eine Matrix mit der nachfolgenden Eigenwertzerlegung.
\begin{equation*}
    H=U\Sigma{}V^H
\end{equation*}    
Wenn H normal ist ($HH^H=H^HH$):
\begin{equation*}
    H=U\Sigma{}U^H=\sum_{i=1}^{n}u_i\lambda_iu_i^H=\color{section}{\sum_{i=1}^{n}\lambda_iu_iu_i^H}\\
\end{equation*} 
Dabei ist
\begin{equation*}
    UU^H = U^HU=I_n \Leftrightarrow \langle{}u_i,u_j\rangle{}=
    \begin{cases}
        1,&i=j\\
        0,&\text{sonst.}
    \end{cases}
\end{equation*}
Wird die Matrix mit einem Eigenvektor "angeregt", so erhalten wir
\begin{align*}
    Hu_\ell&={\color{section}{\left(\sum_{i=1}^n\lambda_iu_iu_i^H\right)}}u_\ell=\sum_{i=1}^n\lambda_iu_i\underbrace{(u_i^Hu_\ell)}_{=\delta_{i-\ell}}\\
    Hu_\ell&=\lambda_\ell{}u_\ell
\end{align*}

Wenn wir den Vektor $x$ als Linearkombination der Basisvektoren $u_i$ schreiben 
\begin{equation*}
    x=\sum_{i=1}^{n}c_iu_i, \text{mit Koeffizienten} c_i=\langle{}x,u_i\rangle
\end{equation*}
Dann erhält man für das Ausgangssignal
\begin{align*}
    Hx& ={\color{section}{\left(\sum_{i=1}^n\lambda_iu_iu_i^H\right)}} \sum_{j=1}^nc_ju_j  \\
    &=\sum_{i=1}^n\sum_{j=1}^n\lambda_ic_ju_i\underbrace{u_i^Hu_j}_{=\delta_{i-j}} \\
    &=\sum_{i=1}^n\lambda_ic_iu_i
\end{align*}
Diese Darstellung entspricht der zuvor abgeleiteten Beziehung für zeitkontinuierliche Systeme
\begin{equation*}
    (Hx)(t)=\int_{-\infty}^\infty\underbrace{\hat{h}(f)}_{\lambda_i}\underbrace{\hat{x}(f)}_{c_i=\langle{}x,u_i\rangle}\underbrace{e^{2\pi ift}}_{u_i}df
\end{equation*}

\subsubsection{Kaskadierung}
\begin{center}
    \begin{tikzpicture}[auto, node distance=2cm,>=latex']
        \node [input, name=rinput] (rinput) {};
        \node [block, right of=rinput] (h1) {$H_1$};
        \node [block, right of=h1] (h2) {$H_2$};
        \node [output, right of=h2, node distance=2cm] (output) {};
        \draw [->] (rinput) -- node{$x(t)$} (h1);
        \draw [->] (h1) -- node{$y_1(t)$} (h2);
        \draw [->] (h2) -- node [name=y] {$y(t)$}(output);
    \end{tikzpicture}
\end{center}
\begin{align*}
    h(t)&=(h_1*h_2)(t)\\
    \hat{h}&=\hat{h_1}\hat{h_2}
\end{align*}

\subsubsection{Weitere spezielle Eingangssignale}
\textbf{Allgemeine Schwingungen:}
\begin{align*}
    x(t)&=e^{st} \\
    \intertext{mit}
    s&=\sigma + i2\pi f_0 \\
    \Re(s) &= \sigma 
    \begin{cases}
        <0, &\text{zeitlich abklingende Einhüllende}\\
        =0, &\text{zeitlich konstante Einhüllende}\\
        >0, &\text{zeitlich anklingende Einhüllende}
    \end{cases}
\end{align*}
Das zugehörige Ausgangssignal lässt sich in diesem Fall schreiben als
\begin{equation*}
    y(t)=(x*h)(t)=\int_{-\infty}^\infty h(\tau)e^{s(t-\tau)}d\tau=e^{st}\underbrace{\int_{-\infty}^\infty h(\tau)e^{-s\tau}d\tau}_{=:H(s)}
\end{equation*}
wobei $H(s)$ die Laplacetransformierte von $h(t)$ genannt wird.